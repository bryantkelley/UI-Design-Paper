\documentclass[sigchi]{acmart}

\title{Linguistics APT-GT}

\author{Tuan Dam}
\affiliation{Auburn University}
\email{tqd0001@auburn.edu}
\author{Amber Jackson}
\affiliation{Auburn University}
\email{asj0017@auburn.edu}
\author{Bryant Kelley}
\affiliation{Auburn University}
\email{bak0016@auburn.edu}
\author{Aditya Mahajan}
\affiliation{Auburn University}
\email{azm0181@auburn.edu}
\author{Jeremy Roberts}
\affiliation{Auburn University}
\email{jhr0022@auburn.edu}
\author{Gabrielle Taylor}
\affiliation{Auburn University}
\email{grt0007@auburn.edu} 
\author{Binh Thanh Tran}
\affiliation{Auburn University}
\email{bzt0016@auburn.edu}

\begin{document}

	\maketitle
	
	\renewcommand{\shortauthors}{Dam et al.}

	\section{Abstract and Introduction}
		The Automatic Phonetic Transcription Grading Tool (APT-GT) is a training application, which allows professors of the Department of Communication Disorders to organize and manage their courses. The primary aim of this application is to provide a training environment for students in the field of communication disorders. This will help students to identify problems in their transcriptions. It also provides an International Phonetic Alphabet (IPA) keyboard, which students can use to insert phonetic symbols. It allows professors to have full control of the course structure which includes uploading audio/video files, managing exams, etc.
	\section{Literature Review}
		\subsection{"Research Papers in Online Learning Performance and Behaviour" by Chia-Wen Tsai}
			This paper was essentially a literature review of various papers about online learning performance. The author provided essential takeaways that can help educators engage their students in online learning. In order for students to have a successful learning experience they must be motivated to learn. Educators should try to integrate other methods of learning and take note on how it affects the students. The teacher can use funny pictures, design online learning activities that encourage the students to collaborate online. They can involve digital games or social network games. These ideas will engage and motivate the students to want to learn and interact. Additionally, students that use OpenCourseware in flipped classrooms are some of the highest scoring students. E-learning is helping students to learn more and achieve higher scores than only the traditional teach styles. One important thing that educators should take into account is the amount of time their students spend online because it can affect their learning outcomes. Students will need to manage their time when taking an online course. Overall this paper encourages professors to think carefully about their students, what will engage them and what would be disadvantageous for them.
		\subsection{"Kindergartners' Conversations in a Computer- Based Technology Classroom" - Eunsook Hyun and Genevieve Davis }
			Over the past two decades, researchers investigating young children and computer technology moved from questioning whether computers can help young children learn to suggesting how educators and parents can best use computers to maximize learning. Much of the earlier research in technology and learning had focused on the use of computers to enhance learners' social, linguistic, and cognitive skills development.
			The paper reveals a study examined emerging inquiries and dialogue of five- to six-year-old kindergartners (9 boys and 9 girls) taking place around computers as they engaged in a mapping project in a technology-rich classroom in the U.S. Discourse analysis of young children's conversations in a technology-rich classroom shed light on their perceptions of computer-based technology as a learning tool. There are five key findings of the paper: (1) cumulative talk patterns among the children evolved into exploratory talk; (2) children's thinking, questioning, and talking was purposeful, reflective, and autonomous; (3) children's speech and dialogue influenced their emergent technological literacy skills; (4) peer collaboration and teacher input scaffolded student development; and (5) students discovered personal preferences in using various tools.
		\subsection{"User Interface Design for E-Learning Software" by Beham F., Mohammad R. Azadehfar and S.D. Katebi}
			User Interface is the means by which the user and a computer system can interact. The success and failure of any software depends on the User Interface Design. Nowadays, UI is the key factor in designing any educational software. Principles and concepts of learning should be considered in addition to UID principles in UID for e-learning. 
			This Paper presents the role of User Interface design in an e-learning application software. According to it, psychology of the student or learner is an important aspect to be considered while developing an e-learning application. As we know UI is point of interaction between user and the education body, so if we missed to implement such correlation then our aims of education may not be achieved.
			There are various means for teaching and learning through electronic devices, such as: computers, the Internet, web, TV, disks, telephone, etc. and the method of using these devices in learning is known as e-learning. However, the paper I referred for my research defines e-learning as: "teaching and learning using computer devices, memories and computer networks in such a way that it focuses on the educational aim, curriculum and provide maximum benefits to the learners".
			UI designer arranges elements (such as: multimedia, and tools like: Textbox, Label etc.) with which user can use computer more easily. Design should begin with an understanding of the intended users, including profiles of their age, sex, physical abilities, education, cultural or ethnic background, motivation, goals and personality. According to the paper, there are 3 golden rules to design UI: (1) Place the user in control; (2) Reduce the user's memory load; (3) Make the interface consistent.
			E-learning applications should be designed in such a way that the pervasive feeling of requirement and motivation grows constantly, and the coercion feeling reduces. Motivation is the key factor to be considered in e-learning systems and its growth, some suggestions to increase motivation are as follows: (1) Using speech interface; (2) Using informal communication style instead of formal; (3) Using variety of colors in educational Medias; (4) Using background music; (5) Learners having control over learning environment.
			Availability is again an important aspect in e-learning system, which means that users can easily access intended content. Allowing learners to access previously taught materials regularly or can look for specific content in the e-learning system anytime. Moreover, whenever the words or phrases that are used in the text exist elsewhere, they should act as a link to navigate, describe, and return the user to the previous page simply.
		\subsection{"Effects of a Conversation-Based Intervention on the Linguistic Skills of Children With Motor Speech Disorders Who Use Augmentative and Alternative Communication" by Gloria Soto and Michael T. Clarke}
			This study was conducted to evaluate the effects of a conversation-based intervention on the expressive vocabulary and grammatical skills of children with severe motor speech disorders and expressive language delay. Many children with severe motor speech disorders (MSDs) are known to experience significant delays in their language development. The study introduces augmentative and alternative communication (AAC) systems, which refers to any form of communication that supplements or replaces natural speech. A multiple probe design across participants was used to examine the effect of a conversation-based intervention on the expressive vocabulary and grammatical skills of eight children with severe MSDs and expressive language delay who use AAC. Eight children aged from 8 to 13 years participated in the study. After a baseline period, a conversation-based intervention was provided for each participant. The targets of the intervention were key linguistic structures essential to early clause formation and grammaticalization, which includes verbs, pronouns, bound grammatical morphemes. During intervention, a student clinician met with each child twice a week. The clinician first presented the child with three photographs depicting the child at three different events, which were provided by the child's parents in the baseline period. The clinician then asked the child to choose a photograph he or she wanted to converse about and describe the event depicted. Upon receiving a response (e.g., "party"), the clinician elicited further information (e.g., using who, where, what questions) and encouraged further communication. The features of the intervention as described above mirror the ethos of dynamic systems theories that propose that learning is a consequence of complex and dynamic interactions between multiple components that must converge at specific levels of intensity for learning to be achieved.
			The findings indicate that, during intervention, all children showed improvement in their production of verbs, pronouns, and bound morphemes. The findings support earlier research demonstrating the effectiveness of adult scaffolding during conversation to increase the production of a range of linguistic structures in children with communication disorders. Children using AAC are frequently described as passive in their interactions with others, as being minimally responsive and reciprocal, and may present with a host of impairments that threaten participation in authentic conversations. The current study established that the use of a format for language intervention that is conversation-based, interactive, structured, and has an expectation for grammaticalization can lead to successful language outcomes.
		\subsection{"Understanding and Using Context" by Anind K. Dey}
			This paper largely looks at what context means, how to define context, and how to use context effectively. The writer sets out that "Context is any information that can be used to characterize the situation of an entity," which successfully details that context is a larger idea than simple breadcrumbs on the user interface. After defining general context, the paper pivots to look into context-aware computing. They begin by referencing another paper's definition of context-aware computing and state that numerous papers following the original paper's publication have too specifically defined the meaning of context.
			The definition of context-aware computing put forth by this paper is "A system is context-aware if it uses context to provide relevant information and/or services to the user, where relevancy depends on the user' s task." This more general definition of context-aware computing is what I see as applicable to our project. Simply put, we need to think about the context that is desired and relevant to what our users do on each page. Additionally, the three categories of features that are listed out by this paper are the presentation of information and services, the automatic execution of a service, and the tagging of context to support retrieval. All three of these categories appear to be important to making the design of the software user friendly and useful.
		\subsection{"Practice makes perfect? The pedagogic value of online independent phonetic transcription practice for speech and language therapy students." by Jill Titterington and Sally Bates}
			The article describes a study performed on a cohort of students studying phonetic transcription and speech disorders. The study involved two parts: giving the students a weekly quiz (the 'Ulster Set') on a specific accent and also giving the students access to an online practice platform (WebFon). At the end of the course, students were to complete a final homework assignment and a questionnaire. WebFon and the 'Ulster Set' both allow students to listen to audio files and respond using phonetic keyboard, the UCL Unicode phonetics keyboard.
			Researchers found a positive correlation between 'Ulster Set' scores and usage of WebFon. In addition, there was a positive correlation between 'Ulster Set' scores and final transcription assignment scores. Students generally found WebFon to be a useful tool as well as thought that the 'Ulster Set' was good preparation for their work in the real world. Overall, the use of these tools had a positive impact on the students' coursework. In addition, students also showed they were much more likely to use WebFon when they were extrinsically motivated, that is, their required postings to the system was many students' primary reason for using the software. However, since they were using the software, they were performing better.
			The authors note that students often struggle more with vowels than consonants. These programs and quizzes should be designed to allow students to practice the topics they need most. With software, its possible to give more regular practice and better enable students to distinguish between accents and disorders.
			Another interesting point of the study was the author's focus on the software side of things. They in order to maximize student engagement, it's necessary for the software to be simple to use. The systems need to be "user friendly and compatible with current IT trends and needs." Future proofing and sustainability are key in building systems of this nature.
	\section{Implementation}
		\subsection{Requirements}
			It's an iterative product and currently version 2.0 is live and in working condition. Our next version will focus on the requirements left out during the last product release. This new release should meet the following criteria:
			\begin{itemize}
				\item Include the following generic features for all roles:
				\begin{itemize}
					\item Reset password
					\item Logout
					\item Edit courses
					\item Display courses
					\item View exam content
					\item View exam results
				\end{itemize}
				\item Include the following specific features for the teacher role:
				\begin{itemize}
					\item Edit exam
					\item Modify/delete assignment
				\end{itemize}
				\item In addition to the above mandatory functionalities, it also requires redesigning of a few webpages.
				\end{itemize}
	\section{Results}
		We have been implementing the the client's written requests and we desire to have a design review with the client in order to solidify mock-ups and address the necessary functionality. In addition, we hope to conduct interviews and usability tests with the client and some of her students while they use the existing system.
	\section{Conclusions}
		\subsection{Progress}
			We have been meeting the deadlines for the deliverables and now we need to physically meet with the client to discuss designs and functionality requirements in order to produce a desirable product for the user.
		\subsection{Client Interaction}
			Client Interaction has been limited, but we desire to have this improve in the near future.
		\subsection{Group Interaction}
			Our group uses our individual strengths to divide and conquer. We then reconvene before and after the sprint ends. Outside of team meetings, we communicate using GroupMe and Trello while maintaining a productive and friendly team dynamic.
	\section{Appendix}
		\subsection{\href{https://doi.org/10.1080/03634520500213397}{"Kindergartners' Conversations in a Computer- Based Technology Classroom" - Eunsook Hyun and Genevieve Davis}}
			This qualitative study examined emerging inquiries and dialogue of five- to six-year-old kindergartners (9 boys and 9 girls) taking place around computers as they engaged in a mapping project in a technology-rich classroom in the U.S. Discourse analysis of young children's conversations in a technology-rich classroom shed light on their perceptions of computer-based technology as a learning tool. Key findings revealed: (a) cumulative talk patterns among the children evolved into exploratory talk; (b) children's thinking, questioning, and talking was purposeful, reflective, and autonomous; (c) children's speech and dialogue influenced their emergent technological literacy skills; (d) peer collaboration and teacher input scaffolded student development; and (e) students discovered personal preferences in using various tools.
		\subsection{\href{https://arxiv.org/ftp/arxiv/papers/1401/1401.6365.pdf}{"User Interface Design for E-Learning Software" by Beham F., Mohammad R. Azadehfar and S.D. Katebi}}
			User interface (UI) is point of interaction between user and computer software. The success and failure of a software application depends on User Interface Design (UID). Possibility of using a software, easily using and learning are issues influenced by UID. The UI is significant in designing of educational software (e-Learning). Principles and concepts of learning should be considered in addition to UID principles in UID for e-learning. In this regard, to specify the logical relationship between education, learning, UID and multimedia at first we readdress the issues raised in previous studies. It is followed by examining the principle concepts of e-learning and UID. Then, we will see how UID contributes to e-learning through the educational software built by authors. Also we show the way of using UI to improve learning and motivating the learners and to improve the time efficiency of using e-learning software.
		\subsection{\href{http://search.ebscohost.com.spot.lib.auburn.edu/login.aspx?direct=true&db=aph&AN=124123725&site=ehost-live}{"Effects of a Conversation-Based Intervention on the Linguistic Skills of Children With Motor Speech Disorders Who Use Augmentative and Alternative Communication" by Gloria Soto and Michael T. Clarke}}
			Purpose: This study was conducted to evaluate the effects of a conversation-based intervention on the expressive vocabulary and grammatical skills of children with severe motor speech disorders and expressive language delay who use augmentative and alternative communication. Method: Eight children aged from 8 to 13 years participated in the study. After a baseline period, a conversation-based intervention was provided for each participant, in which they were supported to learn and use linguistic structures essential for the formation of clauses and the grammaticalization of their utterances, such as pronouns, verbs, and bound morphemes, in the context of personally meaningful and scaffolded conversations with trained clinicians. The conversations were videotaped, transcribed, and analyzed using the Systematic Analysis of Language Transcripts (SALT; Miller \& Chapman, 1991). Results: Results indicate that participants showed improvements in their use of spontaneous clauses, and a greater use of pronouns, verbs, and bound morphemes. These improvements were sustained and generalized to conversations with familiar partners. Conclusion: The results demonstrate the positive effects of the conversation-based intervention for improving the expressive vocabulary and grammatical skills of children with severe motor speech disorders and expressive language delay who use augmentative and alternative communication. Clinical and theoretical implications of conversation-based interventions are discussed and future research needs are identified.
		\subsection{\href{http://citeseer.ist.psu.edu/viewdoc/download?doi=10.1.1.31.9786&rep=rep1&type=pdf}{"Understanding and Using Context" by Anind K. Dey}}
			Context is a poorly used source of information in our computing environments. As a result, we have an impoverished understanding of what context is and how it can be used. In this paper, we provide an operational definition of context and discuss the different ways that context can be used by context-aware applications. We also present the Context Toolkit, an architecture that supports the building of these context-aware applications. We discuss the features and abstractions in the toolkit that make the task of building applications easier. Finally, we introduce a new abstraction, a situation, which we believe will provide additional support to application designers.
		\subsection{\href{http://search.ebscohost.com.spot.lib.auburn.edu/login.aspx?direct=true&db=aph&AN=127784882&site=ehost-live}{"Practice makes perfect? The pedagogic value of online independent phonetic transcription practice for speech and language therapy students." by Jill Titterington and Sally Bates}}
			Accuracy of phonetic transcription is a core skill for speech and language therapists (SLTs) worldwide (Howard \& Heselwood, 2002). The current study investigates the value of weekly independent online phonetic transcription tasks to support development of this skill in year one SLT students. Using a mixed methods observational design, students enrolled in a year one phonetics module completed 10 weekly homework activities in phonetic transcription on a stand-alone tutorial site (WebFon (Bates, Matthews \& Eagles, 2010)) and 5 weekly online quizzes (the 'Ulster Set' (Titterington, unpublished)). Student engagement with WebFon was measured in terms of the number of responses made to 'sparks' on the University's Virtual Learning Environment Discussion Board. Measures of phonetic transcription accuracy were obtained for the 'Ulster Set' and for a stand-alone piece of coursework at the end of the module. Qualitative feedback about experience with the online learning was gathered via questionnaire. A positive significant association was found between student engagement with WebFon and performance in the 'Ulster Set', and between performance in the 'Ulster Set' and final coursework. Students valued both online independent learning resources as each supported different learning needs. However, student compliance with WebFon was significantly lower than with the 'Ulster Set'. Motivators and inhibitors to engagement with the online resources were investigated identifying what best maximised engagement. These results indicate that while 'independent' online learning can support development of phonetic transcription skills, the activities must be carefully managed and constructively aligned to assessment providing the level of valance necessary to ensure effective engagement.
		\subsection{\href{http://www.irrodl.org/index.php/irrodl/article/viewFile/2441/3603}{"Research Papers in Online Learning Performance and Behaviour" by Chia-Wen Tsai}}
\end{document}
  
